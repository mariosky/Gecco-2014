\documentclass{llncs}
\usepackage{subfigure}
\usepackage{amssymb,amsmath}
\sloppy
%
\usepackage{makeidx}  % allows for indexgeneration
\usepackage{graphicx}        % standard LaTeX graphics tool
                             % when including figure files

\usepackage{multicol}        % used for the two-column index
\usepackage[bottom]{footmisc}% places footnotes at page bottom
\usepackage{amsmath,epsfig}
\usepackage{algorithm}
\usepackage{algorithmic}
\usepackage{subfigure}
\usepackage{multirow}
\usepackage{url}


\begin{document} % Something about unreliable workers: assessing the effect, for instance - JJ
\title{There's method in my serendipity:  Using Random Parameters in
  Distributed Evolutionary Algorithms}

% Faltan los autores, ¿no? En este workshop sí se pueden presentar -
% JJ 
\author{Juan J. Merelo Guerv\'os\inst{1} \and Mario Garc\'ia-Valdez\inst{2} \and Leonardo Trujillo\inst{2}  \and Francisco Fern\'andez de Vega\inst{3}}
\institute{Universidad de Granada, Granada, Spain 
\and Instituto Tecnol\'ogico de Tijuana, Tijuana BC, Mexico
\and Grupo de Evoluci\'on Artificial, Universidad de Extremadura, M\'erida, Spain
\email{jmerelo@geneura.ugr.es, mario@tectijuana.edu.mx,}\\
\email{leonardo.trujillo@tectijuana.edu.mx, fcofdez@unex.es}}

\authorrunning{Garc\'ia-Valdez et al.}

\maketitle              % typeset the title of the contribution

\keywords{Distributed evolutionary algorithms, parameter setting}

% While EAs are inspired by evolution, they mostly follow an abstract model of the natural process.
% For instance, one aspect that is omitted from most EAs is an open-ended search process, in practice EAs are used to solve problems with well defined objectives, while natural evolution is an adaptive process without an a priori goal or purpose. Such open-ended EAs have been developed, mostly for artistic design \cite{Musart}, interactive evolution \cite{ie1} and artificial life systems \cite{avida}. Other interesting features of biological evolution, is that it is an intrinsically parallel, distributed and asynchronous process, undoubtedly important features that have allowed evolution to produce impressive results throughout nature. However, some of these features are not trivially included into standard EAs, which are mostly coded as sequential and synchronous algorithms \cite{eiben}.

% This intro is rather convoluted, and does not really introduce the
% topic. Any new idea? - JJ 

% Why not start with a problem statement about parameter tuning in distributed and asynchronous EAs?, borrowing from the workshop description: 

% "All too often heuristics and meta-heuristics for combinatorial optimisation problems require significant parameter tuning to work most effectively. Often this tuning is performed without any a priori knowledge as to how good values of parameters might depend on features of the problem. This lack of knowledge can lead to lot of computational effort and also has the danger of being limited to only problem instances that are similar to those that have been seen before."    

% We could start the Intro from:
\section{Introduction} % Bueno, no hay más secciones, pero si no queda
                       % muy pegado arriba - JJ
Several Evolutionary Algorithms (EA) that distribute the evolutionary
process among heterogeneous devices have been proposed, from
controlled nodes in a in-house cluster or grid to those outside the
data center, in users’ web browsers and smart phones or external cloud
based virtual machines. This reach out approach allows researchers the
use of low cost computational power that would not be available
otherwise, but on the other hand, have the challenge to manage
heterogeneous unreliable computing resources. Lost connections, low
bandwidth communications, abandoned work, security and privacy issues
are all common in these settings. 

%For instance, a large body of work exists in EA parallelization, using multiple CPU cores, multiple nodes and GPUs.
%However, distributed and asynchronous EAs have started to become common only recently \cite{sofea:naco}. %cite - JJ In particular, recent trends in information technology have opened new lines of future development for EC research. Today, computing resources computing power range from personal computers and smart-devices to massive data centers. These resources are easily accessible through popular Internet technologies, such as cloud computing \cite{varia2008cloud} , peer-to-peer (P2P) , and web environments. % references - JJ

%These technologies are intended for the development of parallel, distributed and asynchronous systems, so that an EA developed on top of them could easily reap the benefits of these features.
%As stated before, several EAs have been proposed that distribute  the evolutionary process among heterogeneous devices, not only among controlled
%nodes within an in-house cluster or grid, but also to others outside the data center, in users' web browsers, smart phones or external cloud based virtual machines.
%This reach out approach allows researchers to use low cost computational 
%power that would not be available otherwise, but on the other hand, have the 
%challenge to manage heterogeneous unreliable computing resources. 

In particular, we are interested in EAs systems that follow a
pool-based approach, where the search process is conducted by a
collection of possibly heterogeneous processes that collaborate using
a shared repository or population pool. We will refer to such
algorithms as Pool-based EAs or PEAs, and highlight the fact that such
systems are intrinsically parallel, distributed and asynchronous. 

%Despite promising results, PEAs present several notable challenges. From a technological perspective, for example, lost 
%connections, low bandwidth, abandoned work, security and privacy are all important issues that must be addressed. 
%However, 
Here we focus on a common issue with most EAs, that is only amplified in a PEA, a problem that can be referred to as parametrization.
In general, among machine learning methodologies, EAs are highly criticized by the large number of parameters they possess,
that for real world problems need to be tuned empirically or require additional heuristic processes to be included into the search to
adjust the parameters automatically \cite{ss}.
In the case of a PEAs, this issue is magnified since the underlying system architecture adds several degrees of freedom to the search process.

Our  work has been focused on a recently proposed pool-based system called EvoSpace, a framework to develop PEAs using 
a heterogeneous collection of possibly unreliable resources.
EvoSpace is based on Linda's tuple space coordination model, where each node asynchronously
pulls its work from a central shared memory  \cite{Evospace,Musart,FreeLunch,Fire}. The core elements of EvoSpace are a central 
repository for the evolving population and remote clients here called EvoWorkers
which pull random samples of the population to perform on them the basic evolutionary
processes (selection, variation and survival), once the work is done, the
modified sample is pushed back to the central population.
Despite some promising initial results \cite{Evospace,Musart,FreeLunch,Fire}, research devoted to EvoSpace has not addressed the parametrization issue mentioned above.
In the study presented here, a recent approach called Randomized Parameter Setting Strategy (RPSS) \cite{fuku1,fuku2} is applied to EvoSpace and tested on several
benchmark problems.
The idea behind RPSS is that in a distributed EA, algorithm parametrization may be completely skipped for a successful search,
with research showing that when the number of distributed process is large enough, algorithm parameters can be set randomly and still achieve
good overall results.
However, work on RPSS has only focused on the well-known Island Model for EAs, a distributed but synchronous system.
% En el paper le llaman Distributed Island Model, para diferenciarla por ejemplo de Parallel Island Model en una maquina multiprocesador -Mario
% Pero, creo que lo distribuido no cambia el modelo, solo la implementacion, en el otro articulo si se refieren a el como Island Model ... tal vez si le ponemos (distibuted) Island Model ?
On the other hand, the goal of this line of work is to evaluate RPSS on a
complete PEA implemented through EvoSpace.

Using randomized parameter settings falls within the area of
self-parametrization of (distributed) evolutionary algorithms. So far
we have proved that a randomized parameter evolutionary algorithm can
obtain decent results, at least comparable to the best homogenized
non-random setting. However, a experimental or heuristic setting would
probably yield better results. Is there a way to adapt parameters to
the particular characteristics of each computing node? That is an
issue we would like to raise, and discuss, in this workshop. 


\section{Acknowledgments*}

This work has been supported in part by project ANYSELF
(TIN2011-28627-C04-02). 

%
% ---- Bibliography ----
% There are problems with the bibliography codification.
\bibliographystyle{abbrv}
\bibliography{biblio}

\end{document}
