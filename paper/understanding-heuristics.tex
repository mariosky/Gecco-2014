\documentclass{llncs}
\usepackage{subfigure}
\usepackage{amssymb,amsmath}
\sloppy
%
\usepackage{makeidx}  % allows for indexgeneration
\usepackage{graphicx}        % standard LaTeX graphics tool
                             % when including figure files

\usepackage{multicol}        % used for the two-column index
\usepackage[bottom]{footmisc}% places footnotes at page bottom
\usepackage{amsmath,epsfig}
\usepackage{algorithm}
\usepackage{algorithmic}
\usepackage{subfigure}
\usepackage{multirow}
\usepackage{url}


\begin{document} % Something about unreliable workers: assessing the effect, for instance - JJ
\title{ Randomized Parameter Settings for Heterogeneous Workers in a Pool-Based Evolutionary Algorithm}

\maketitle              % typeset the title of the contribution

\keywords{Distributed evolutionary algorithms, parameter setting}

Thanks in part  to evolutionary computation (EC) research, scientists and engineers from many fields now understand the remarkable power of
the natural search process described by the biological theory of Neo-Darwinian evolution \cite{holland}. 
Inspired in biological evolution, EC researchers have developed a variety of search and optimization algorithms \cite{eiben}.
%Empty statement, maybe delete.

While EAs are inspired by evolution, they mostly follow an abstract model of the natural process.
For instance, one aspect that is omitted from most EAs is an open-ended search process,
in practice EAs are used to solve problems with well defined objectives, while natural evolution is an adaptive process without an a priori goal or purpose.
Such open-ended EAs have been developed, mostly for artistic design \cite{Musart}, interactive evolution \cite{ie1} and artificial life systems \cite{avida}.
Other interesting features of biological evolution, is that it is an intrinsically parallel, distributed and asynchronous process,
undoubtedly important features that have allowed evolution to produce impressive results throughout nature.
However, some of these features are not trivially included into standard EAs,
which are mostly coded as sequential and synchronous algorithms \cite{eiben}.

For instance, a large body of work exists in EA parallelization, using multiple CPU cores, multiple nodes and GPUs.
However, distributed and asynchronous EAs have started to become common only recently.
In particular, recent trends in information technology have opened new lines of future development for EC research.
Today, computing resources computing power range from personal computers and smart-devices to massive data centers.
These resources are easily accessible through popular Internet technologies, such as cloud computing, 
peer-to-peer (P2P), and web environments.

Moreover, these technologies are intended for the development of parallel, distributed and asynchronous systems,
such that an EA developed on top of them could easily reap the benefits of these features.
As stated before, several EAs have been proposed that distribute  the evolutionary process among heterogeneous devices, not only among controlled
nodes within an in-house cluster or grid, but also to others outside the data center, in users' web browsers, smart phones or external cloud based virtual machines.
This reach out approach allows researchers to use low cost computational 
power that would not be available otherwise, but on the other hand, have the 
challenge to manage heterogeneous unreliable computing resources. 
In particular, we are interested in systems that follow a pool-based approach, where the search process is conducted
by a collection, of possibly heterogeneous, collaborating processes using a shared repository or population pool.
We will refer to such algorithms as Pool-based EAs or PEAs, and highlight the fact that such systems are
intrinsically parallel, distributed and asynchronous.

Despite promising results, PEAs present several notable challenges. From a technological perspective, for example, lost 
connections, low bandwidth, abandoned work, security and privacy are all important issues that must be addressed. 
However, here we focus on a common issue with most EAs, that is only amplified in a PEA, a problem that can be referred to as parametrization.
In general, among machine learning methodologies, EAs are highly criticized by the large number of parameters they posses,
that for real world problems need to be tuned empirically or require additional heuristic processes to be included into the search to
adjust the parameters automatically \cite{ss}.
In the case of a PEAs, this issue is magnified since the underlying system architecture adds several degrees of freedom to the search process.

This work studies a recently proposed pool-based system called EvoSpace, a framework to develop PEAs using 
a heterogeneous collection of possibly unreliable resources.
EvoSpace is based on Linda's tuple space coordination model, where each node asynchronously
pulls its work from a central shared memory  \cite{Evospace,Musart,FreeLunch,Fire}. The core elements of EvoSpace are a central 
repository for the evolving population and remote clients here called EvoWorkers
which pull random samples of the population to perform on them the basic evolutionary
processes (selection, variation and survival), once the work is done, the
modified sample is pushed back to the central population.
Despite some promising initial results \cite{Evospace,Musart,FreeLunch,Fire}, research devoted to EvoSpace has not addressed the parametrization issue mentioned above.
In the study presented here, a recent approach called Randomized Parameter Setting Strategy (RPSS) \cite{fuku1,fuku2} is applied to EvoSpace and tested on several
benchmark problems.
The idea behind RPSS is that in a distributed EA, algorithm parametrization may be completely skipped for a successful search,
with research showing that when the number of distributed process is large enough, algorithm parameters can be set randomly and still achieve
good overall results.
However, work on RPSS has only focused on the well-known Island Model for EAs, a distributed but synchronous system.
% En el paper le llaman Distributed Island Model, para diferenciarla por ejemplo de Parallel Island Model en una maquina multiprocesador -Mario
% Pero, creo que lo distribuido no cambia el modelo, solo la implementacion, en el otro articulo si se refieren a el como Island Model ... tal vez si le ponemos (distibuted) Island Model ?
On the other hand, the goal of this work is to evaluate RPSS on a complete PEA implemented through EvoSpace.



\section{Acknowledgments}

This work has been supported in part by project ANYSELF
(TIN2011-28627-C04-02). 

%
% ---- Bibliography ----
% There are problems with the bibliography codification.
\bibliographystyle{abbrv}
\bibliography{biblio}

\end{document}
